\definecolor{dkgreen}{rgb}{0,0.6,0}
\definecolor{gray}{rgb}{0.5,0.5,0.5}
\definecolor{mauve}{rgb}{0.58,0.0,0.82}
\lstset{
    language=[x86masm]Assembler,
    %language=C,
    basicstyle=\ttfamily\footnotesize,
    %numbers=left, 
    %numberstyle=\tiny, 
    %stepnumber=2, 
    %numbersep=5pt,
    backgroundcolor=\color{white},
    showspaces=false,
    showstringspaces=false,
    showtabs=false,
    %frame=single,
    frame=trBL,
    rulecolor=\color{black},
    tabsize=4,                      % sets default tabsize to 2 spaces
    captionpos=b,                   % sets the caption-position to bottom
    breaklines=true,                % sets automatic line breaking
    breakatwhitespace=false,        % sets if automatic breaks should only 
                                    % happen at whitespace
    title=\lstname,                 % show the filename of files included 
                                    % with \lstinputlisting;
                                    % also try caption instead of title
    keywordstyle=\color{blue},      % keyword style
    commentstyle=\color{dkgreen},   % comment style
    stringstyle=\color{mauve},      % string literal style
    morekeywords={*,__asm}          % if u want to add more keywords to the set
    %escapeinside={\%*}{*)},        % if you want to add a comment within 
}

\lstdefinestyle{makefile}
{
    numberblanklines=false,
    language=make,
    basicstyle=\ttfamily, %\footnotesize,
    tabsize=8,
    keywordstyle=\color{red},
    backgroundcolor=\color{white},
    showspaces=false,
    showstringspaces=false,
    showtabs=false,
    backgroundcolor=\color{lightlightgray}, % Choose background color
    frame=single,                   % A frame around the code
    %frame=none,
    %frame=trBL,
    rulecolor=\color{black},
    captionpos=b,                   % sets the caption-position to bottom
    breaklines=true,                % sets automatic line breaking
    breakatwhitespace=false,        % sets if automatic breaks should only 
                                    % happen at whitespace
    %title=\lstname,                 % show the filename of files included 
                                    % with \lstinputlisting;
                                    % also try caption instead of title
    keywordstyle=\color{blue},      % keyword style
    commentstyle=\color{dkgreen},   % comment style
    stringstyle=\color{mauve},      % string literal style
    morekeywords={*,__asm},         % if u want to add more keywords to the set
    %escapeinside={\%*}{*)},        % if you want to add a comment within 
    identifierstyle=                %plain identifiers for make
}
\definecolor{Brown}{cmyk}{0,0.81,1,0.60}
\definecolor{OliveGreen}{cmyk}{0.64,0,0.95,0.40}
\definecolor{CadetBlue}{cmyk}{0.62,0.57,0.23,0}
\definecolor{lightlightgray}{gray}{0.9}
\definecolor{dkgreen}{rgb}{0,0.6,0}
\definecolor{gray}{rgb}{0.5,0.5,0.5}
\definecolor{mauve}{rgb}{0.58,0,0.82}
\lstset{
language=C,                             % Code langugage
basicstyle=\ttfamily\footnotesize,      % Code font, Examples: \footnotesize, \ttfamily
keywordstyle=\color{blue},              % Keywords font ('*' = uppercase)
commentstyle=\color{dkgreen},           % Comments font
stringstyle=\color{mauve},              % string literal font
escapeinside={\%*}{*},                  % if you want to add a comment within your code
numbers=left,                           % Line nums position
numberstyle=\tiny,                      % Line-numbers fonts
stepnumber=0,                           % Step between two line-numbers
numbersep=5pt,                          % How far are line-numbers from code
backgroundcolor=\color{lightlightgray}, % Choose background color
% frame=none,                             % A frame around the code
frame=single,
tabsize=2,                              % Default tab size
captionpos=b,                           % Caption-position = bottom
breaklines=true,                        % Automatic line breaking?
breakatwhitespace=false,                % Automatic breaks only at whitespace?
showspaces=false,                       % Dont make spaces visible
showtabs=false,                         % Dont make tabls visible
columns=flexible,                       % Column format
morekeywords={__global__, __device__},  % CUDA specific keywords
}
\lstdefinestyle{asm1} {
language=[x86masm]Assembler,            % Code langugage
basicstyle=\ttfamily,                   % Code font, Examples: \footnotesize, \ttfamily
keywordstyle=\color{blue},              % Keywords font ('*' = uppercase)
commentstyle=\color{dkgreen},           % Comments font
stringstyle=\color{mauve},              % string literal font
escapeinside={\%*}{*},                  % if you want to add a comment within your code
numbers=left,                           % Line nums position
numberstyle=\tiny,                      % Line-numbers fonts
stepnumber=0,                           % Step between two line-numbers
numbersep=5pt,                          % How far are line-numbers from code
backgroundcolor=\color{lightlightgray}, % Choose background color
%frame=none,                             % A frame around the code
frame=single,
tabsize=2,                              % Default tab size
captionpos=b,                           % Caption-position = bottom
breaklines=true,                        % Automatic line breaking?
breakatwhitespace=false,                % Automatic breaks only at whitespace?
showspaces=false,                       % Dont make spaces visible
showtabs=false,                         % Dont make tabls visible
columns=flexible,                       % Column format
morekeywords={__global__, __device__},  % CUDA specific keywords
}

\lstdefinestyle{bash}
{
    numberblanklines=false,
    language=bash,
    basicstyle=\ttfamily\color{white}\bfseries, %\footnotesize,
    tabsize=8,
    keywordstyle=\color{red},
    backgroundcolor=\color{black},
    %backgroundcolor=\color{lightlightgray}, % Choose background color
    showspaces=false,
    showstringspaces=false,
    showtabs=false,
    %frame=single,                   % A frame around the code
    frame=none,
    %frame=trBL,
    rulecolor=\color{black},
    captionpos=b,                   % sets the caption-position to bottom
    breaklines=true,                % sets automatic line breaking
    breakatwhitespace=false,        % sets if automatic breaks should only 
                                    % happen at whitespace
    %title=\lstname,                 % show the filename of files included 
                                    % with \lstinputlisting;
                                    % also try caption instead of title
    keywordstyle=\color{blue},      % keyword style
    commentstyle=\color{dkgreen},   % comment style
    stringstyle=\color{mauve},      % string literal style
    morekeywords={*,__asm},         % if u want to add more keywords to the set
    %escapeinside={\%*}{*)},        % if you want to add a comment within 
    identifierstyle= %plain identifiers for make
}

\lstdefinestyle{asm}
{
    numberblanklines=false,
    language=[x86masm]Assembler,
    basicstyle=\ttfamily\footnotesize,
    tabsize=8,
    keywordstyle=\color{red},
    backgroundcolor=\color{lightlightgray}, % Choose background color
    showspaces=false,
    %showspaces=true,
    showstringspaces=false,
    showtabs=false,
    frame=single,                   % A frame around the code
    %frame=none,
    %frame=trBL,
    rulecolor=\color{black},
    captionpos=b,                   % sets the caption-position to bottom
    breaklines=true,                % sets automatic line breaking
    breakatwhitespace=false,        % sets if automatic breaks should only 
                                    % happen at whitespace
    %title=\lstname,                 % show the filename of files included 
                                    % with \lstinputlisting;
                                    % also try caption instead of title
    keywordstyle=\color{blue},      % keyword style
    commentstyle=\color{dkgreen},   % comment style
    stringstyle=\color{mauve},      % string literal style
    morekeywords={*,__asm},         % if u want to add more keywords to the set
    %morecomment=[l]{//},
    %escapeinside={\%*}{*)},        % if you want to add a comment within 
    identifierstyle= %plain identifiers for make
}


%%% Local Variables:
%%% mode: latex
%%% TeX-master: "main_book"
%%% End:
